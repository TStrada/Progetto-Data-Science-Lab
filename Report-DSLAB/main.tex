%%%%%%%%%%%%%%%%%%%%%%%%%%%%%%%%%%%%%%%%%
% Wenneker Article
% LaTeX Template
% Version 2.0 (28/2/17)
%
% This template was downloaded from:
% http://www.LaTeXTemplates.com
%
% Authors:
% Vel (vel@LaTeXTemplates.com)
% Frits Wenneker
%
% License:
% CC BY-NC-SA 3.0 (http://creativecommons.org/licenses/by-nc-sa/3.0/)
%
%%%%%%%%%%%%%%%%%%%%%%%%%%%%%%%%%%%%%%%%%

%----------------------------------------------------------------------------------------
%	PACKAGES AND OTHER DOCUMENT CONFIGURATIONS
%----------------------------------------------------------------------------------------

\documentclass[12pt, a4paper, twocolumn]{article} % 10pt font size (11 and 12 also possible), A4 paper (letterpaper for US letter) and two column layout (remove for one column)
\usepackage{setspace}
\usepackage{multirow}
\usepackage{float}
\usepackage{hyperref}
\usepackage[USenglish,UKenglish,french,spanish,italian]{babel}
\usepackage{graphicx}
\graphicspath{ {./Immagini/} }

\input{structure.tex} % Specifies the document structure and loads requires packages

%----------------------------------------------------------------------------------------
%	ARTICLE INFORMATION
%----------------------------------------------------------------------------------------

\title{Data Science Lab: Vendite E-commerce} % The article title

\author{
	Davide Abete x\\
	Fabrizio Cominetti x\\
	Ruben Agazzi x\\
	Tommaso Strada x\\
	Alessandro Fasani x % Authors
}

%----------------------------------------------------------------------------------------

\begin{document}

\selectlanguage{italian}

% \maketitle % Print the title

\thispagestyle{firstpage} % Apply the page style for the first page (no headers and footers)

%----------------------------------------------------------------------------------------
%	ABSTRACT
%----------------------------------------------------------------------------------------
\twocolumn[
  \begin{@twocolumnfalse}
    \maketitle
    \begin{abstract}\\
      \noindent{L'obiettivo principale del progetto è quello di valutare e selezionare il miglior modello predittivo relativamente alle stime delle vendite di alcuni settori, precisamente Pesca, Calcio e x, di un attività di e-commerce.}\\
      Il periodo preso in considerazione va dal giorno x al giorno x.\\
      Per un attività commerciale presente sulla rete è di fondamentale importanza prevedere le vendite che saranno effettuate nei periodi successivi per ogni settore a disposizione, in modo tale da organizzare disponibilità di magazzino e di spedizione.\\
      I dati sono stati modellati in funzione dell'obiettivo e aggregati secondo diverse granularità, ovvero con frequenza annuale, trimestrale, mensile e settimanale, in modo tale da valutare le performance dei modelli presi in considerazione.\\
      Come sarà possibile verificare nel proseguio del report, la granularità più efficace, relativamente agli obiettivi preposti dal progetto, è risultata quella x.\\
      Il progetto è stato effettuato prendendo in considerazione i tre settori con maggiore disponibilità effettiva di osservazioni, ma i modelli utilizzati possono essere applicati anche ad altri settori, per realizzare in questo modo una panoramica completo sull'intero e-commerce.
    \end{abstract}
  \end{@twocolumnfalse}
]
\bigskip

\clearpage
% \lettrineabstract{x}
\bigskip
\tableofcontents

%----------------------------------------------------------------------------------------
%	ARTICLE CONTENTS
%----------------------------------------------------------------------------------------

\section{Introduzione}
La realizzazione di questo progetto ha visto come obiettivo principale quello di testare, valutare e decretare il miglior modello predittivo al fine di stimare le vendite in euro realizzate dai vari settori all'interno di un e-commerce.\\
Prima di tutto, i dati sono stati pre-processati e 'puliti', in modo tale da renderli efficaci e ottimali rispetto allo scopo del progetto. I modelli predittivi, infatti, richiedevano un certo tipo di modellazione dei dati in input per utilizzarli al proprio interno e produrre delle previsioni.\\
Una volta effettuato il pre-processing dei dati, il focus è passato sul test dei vari modelli, ognuno con le diverse granularità scelte in fase di programmazione.\\
x\\
Periodo scelto e caratteristiche periodo temporale.

\subsection{Punti Principali}
\begin{itemize}
	\item Vendite 
	\item Settori 
	\item Previsione
\end{itemize}

\subsection*{Heading on level 2}
Lorem ipsum dolor sit amet, consectetuer adipiscing elit. Aenean commodo ligula eget dolor. Aenean massa. Cum sociis natoque penatibus et magnis dis parturient montes, nascetur ridiculus mus. Donec quam felis, ultricies nec, pellentesque eu, pretium quis, sem.
\bigskip

\begin{itemize}
	\item First item in a list 
	\item Second item in a list 
	\item Third item in a list
\end{itemize}
Lorem ipsum dolor sit amet, consectetuer adipiscing elit. Aenean commodo ligula eget dolor. Aenean massa. Cum sociis natoque penatibus et magnis dis parturient montes, nascetur ridiculus mus. Donec quam felis, ultricies nec, pellentesque eu, pretium quis, sem. Nulla consequat massa quis enim. 

Donec pede justo, fringilla vel, aliquet nec, vulputate eget, arcu. In enim justo, rhoncus ut, imperdiet a, venenatis vitae, justo. Nullam dictum felis eu pede mollis pretium. Integer tincidunt. Cras dapibus. Vivamus elementum semper nisi. Aenean vulputate eleifend tellus. Aenean leo ligula, porttitor eu, consequat vitae, eleifend ac, enim. Aliquam lorem ante, dapibus in, viverra quis, feugiat a, tellus. Phasellus viverra nulla ut metus varius laoreet. Quisque rutrum. Aenean imperdiet. Etiam ultricies nisi vel augue. Curabitur ullamcorper ultricies 

\section{Obiettivo}
L'obiettivo del progetto è quello di identificare il miglior modello predittivo da fornire ai gestori dell'e-commerce in questione.\\
Conoscere la previsione relativa agli incassi (in euro) dei vari settori all'interno del sito potrebbe infatti essere utile a diversi scopi all'interno dell'azienda, come ad esempio in fase di programmazione e ordini di materiale, ma anche per avere un'idea di quando applicare o meno determinati sconti e promozioni ai vari settori, con il fine di raggiungere gli obiettivi di vendita prefissati.\\
Periodo\\
Realizzazione di dashboard\\
Destinatari del progetto.

\begin{table}
\caption{Random table}
\centering
	\begin{tabular}{llr}
		\toprule
		\multicolumn{2}{c}{Name} \\
		\cmidrule(r){1-2}
			First name & Last Name & Grade \\
		\midrule
			John & Doe & $7.5$ \\
			Richard & Miles & $2$ \\
		\bottomrule
	\end{tabular}
\end{table}

\section{Aspetti Metodologici}
I modelli selezionati ed utilizzati all'interno del progetto sono i seguenti: ARIMA, TBATS, PROPHET, XGBOOST.\\
Di seguito osserviamo i modelli in modo più approfondito da un punto di vista teorico.

\subsection*{ARIMA}
In statistica per modello ARIMA (acronimo di AutoRegressive Integrated Moving Average) si intende una particolare tipologia di modelli atti ad indagare serie storiche che presentano caratteristiche particolari. Fa parte della famiglia dei processi lineari non stazionari.\\
Un modello ARIMA(p,d,q) deriva da un modello ARMA(p,q) a cui sono state applicate le differenze di ordine d per renderlo stazionario. In caso di stagionalità nei dati si parla di modelli SARIMA o ARIMA(p,d,q)(P,D,Q).\\
Dunque, il modello ARIMA nasce aggiungendo l'integrazione (I) alla combinazione dei modelli autoregressivo (AR) e a media mobile (MA). Il modello ARIMA è composto dalle seguenti componenti:

\begin{description}
	\item[P] l'ordine della componente autoregressiva
	\item[D] grado della differenziazione
	\item[Q] ordine della componente a media mobile
\end{description}

An integrated ARMA model of order d is a stochastic process that becomes stationary after differentiating it d times.\\
\\
$modello$\\
\\
Dove:
\begin{description}
	\item[x] x
	\item[x] x
	\item[x] x
\end{description}

\subsection*{TBATS}
Il modello TBATS è in grado di considerare e lavorare con stagionalità multiple e complesse.

\includegraphics{tbats.png}
\\
sopra img da dispensa, sotto da report\\
\includegraphics{tbats2.png}

\begin{description}
	\item[First] This is the first item
	\item[Last] This is the last item
\end{description}
Nullam dictum felis eu pede mollis pretium. Integer tincidunt. Cras dapibus. Vivamus elementum semper nisi. Aenean vulputate eleifend tellus. Aenean leo ligula, porttitor eu, consequat vitae, eleifend ac, enim. Aliquam lorem ante, dapibus in, viverra quis, feugiat a, tellus. Phasellus viverra nulla ut metus varius laoreet. Quisque rutrum. Aenean imperdiet. Etiam ultricies nisi vel augue. Curabitur ullamcorper ultricies

\subsection*{PROPHET}
Prophet è invece una procedura dedita alla previsione di serie storiche basata su modelli addittivi, dove i trend non lineari sono analizzati con diverse stagionalità.\\
Il modello decompone la serie storica in trend, stagionalità e festività.

\begin{description}
	\item[First] This is the first item
	\item[Last] This is the last item
\end{description}
Prophet is a procedure for forecasting time series data based on an additive model where non-linear trends are fit with yearly, weekly, and daily seasonality, plus holiday effects.\\
It works best with time series that have strong seasonal effects and several seasons of historical data. Prophet is robust to missing data and shifts in the trend, and typically handles outliers well.\\
The procedure makes use of a decomposable time series model with three main model components: trend, seasonality, and holidays.\\
Similar to a generalized additive model (GAM), with time as a regressor, Prophet fits several linear and non-linear functions of time as components.\\ In its simplest form:
\\
$y(t) = g(t) + s(t) + h(t) + e(t)$
\\
dove:
\begin{description}
	\item[g(t)] trend models non-periodic changes (i.e. growth over time)
	\item[s(t)] seasonality presents periodic changes (i.e. weekly, monthly, yearly)
	\item[h(t)] ties in effects of holidays (on potentially irregular schedules $\geq$ 1 day(s))
	\item[e(t)] covers idiosyncratic changes not accommodated by the model
\end{description}
\cite{mathprophet}\cite{fbprophet}

In other words, the procedure’s equation can be written:
\includegraphics{prophet.png}
\\
Prophet is essentially “framing the forecasting problem as a curve-fitting exercise” rather than looking explicitly at the time based dependence of each observation.\\
The procedure provides two possible trend models for g(t), “a saturating growth model, and a piecewise linear model.”\\
The seasonal component s(t) provides a adaptability to the model by allowing periodic changes based on sub-daily, daily, weekly and yearly seasonality.\\
Prophet relies on Fourier series to provide a malleable model of periodic effects. P is the regular period the time series will have (e.g. P = 365.25 for yearly data or P = 7 for weekly data, when time is scaled in days).\\
Impact of a particular holiday on the time series is often similar year after year, making it an important incorporation into the forecast. The component h(t) speaks for predictable events of the year including those on irregular schedules (e.g. Black Friday or the Superbowl). To utilize this feature, the user needs to provide a custom list of events. Fusing this list of holidays into the model is made straightforward by assuming that the effects of holidays are independent.\\

\subsection*{XGBOOST}
XGBoost is an efficient implementation of gradient boosting for classification and regression problems.\\
XGBoost can also be used for time series forecasting, although it requires that the time series dataset be transformed into a supervised learning problem first. It also requires the use of a specialized technique for evaluating the model called walk-forward validation, as evaluating the model using k-fold cross validation would result in optimistically biased results.\\
\\
XGBRegressor uses a number of gradient boosted trees (referred to as n_estimators in the model) to predict the value of a dependent variable. This is done through combining decision trees (which individually are weak learners) to form a combined strong learner.\\
When forecasting a time series, the model uses what is known as a lookback period to forecast for a number of steps forward. For instance, if a lookback period of 1 is used, then the X_train (or independent variable) uses lagged values of the time series regressed against the time series at time t (Y_train) in order to forecast future values.\\
\cite{xgboost}

\begin{description}
	\item[First] This is the first item
	\item[Last] This is the last item
\end{description}
Nullam dictum felis eu pede mollis pretium. Integer tincidunt. Cras dapibus. Vivamus elementum semper nisi. Aenean vulputate eleifend tellus. Aenean leo ligula, porttitor eu, consequat vitae, eleifend ac, enim. Aliquam lorem ante, dapibus in, viverra quis, feugiat a, tellus. Phasellus viverra nulla ut metus varius laoreet. Quisque rutrum. Aenean imperdiet. Etiam ultricies nisi vel augue. Curabitur ullamcorper ultricies

\section{Dati}
Il dataset utilizzato per il progetto è il dataset "serie-storiche-ecommerce" ed è un file di tipo CSV (Comma Separated Values).\\
Il file si presentava con un problema relativo alla divisione dell'importo in euro in due differenti colonne, è stata perciò effettuata una correzione per unire le due colonne citate in un unica colonna.\\
Considerando la correzione effettuata, all'interno del file sono presenti le seguenti colonne:
\begin{description}
	\item[data] contenente la data di rilevazione nel seguente formato: DD/MM/YYYY
	\item[totale] importo in euro dell'incasso di uno specifico settore in quel giorno
	\item[settore] testo che identifica il settore dell'e-commerce di riferimento
\end{description}
Per ciascun settore è dunque presente il totale delle vendite (in euro) effettuate in quella data. Le rilevazioni sono dunque giornaliere e divise per settore. Sono pochi i settori che presentano una fetta consistente di rilevazioni, al contrario, per molti settori il numero di osservazioni è limitato.\\
% magari aggiungere immagini relative a numero di osservazioni dei settori
Per questo motivo le analisi successive saranno effettuate considerando i settori con il maggior numero di osservazioni presenti.\\
I dati a disposizione coprono il periodo compreso tra il 2 febbraio 2013 e l'8 aprile 2022.\\
Il file iniziale è composto da un totale di 25262 righe e dalle 3 colonne descritte sopra.

\subsection{Manipolazione Dati}
Lorem ipsum dolor sit amet, consectetuer adipiscing elit. Aenean commodo ligula eget dolor. Aenean massa. Cum sociis natoque penatibus et magnis dis parturient montes, nascetur ridiculus mus. Donec quam felis, ultricies nec, pellentesque eu, pretium quis, sem. 

\section{Analisi Pesca}
Lorem ipsum dolor sit amet, consectetuer adipiscing elit. Aenean commodo ligula eget dolor. Aenean massa. Cum sociis natoque penatibus et magnis dis parturient montes, nascetur ridiculus mus. Donec quam felis, ultricies nec, pellentesque eu, pretium quis, sem. Nulla consequat massa quis enim. Donec pede justo, fringilla vel, aliquet nec, vulputate eget, arcu. In enim justo, rhoncus ut, imperdiet a, venenatis vitae, justo. Nullam dictum felis eu pede mollis pretium. Integer tincidunt. Cras dapibus. Vivamus elementum semper nisi. Aenean vulputate eleifend tellus. Aenean leo ligula, porttitor eu, consequat vitae, eleifend ac, enim. Aliquam lorem ante, dapibus in, viverra quis, feugiat a, tellus. Phasellus viverra nulla ut metus varius laoreet. Quisque rutrum. Aenean imperdiet. Etiam ultricies nisi vel augue. Curabitur ullamcorper ultricies

\section{Analisi Calcio}
Lorem ipsum dolor sit amet, consectetuer adipiscing elit. Aenean commodo ligula eget dolor. Aenean massa. Cum sociis natoque penatibus et magnis dis parturient montes, nascetur ridiculus mus. Donec quam felis, ultricies nec, pellentesque eu, pretium quis, sem. Nulla consequat massa quis enim. Donec pede justo, fringilla vel, aliquet nec, vulputate eget, arcu. In enim justo, rhoncus ut, imperdiet a, venenatis vitae, justo. Nullam dictum felis eu pede mollis pretium. Integer tincidunt. Cras dapibus. Vivamus elementum semper nisi. Aenean vulputate eleifend tellus. Aenean leo ligula, porttitor eu, consequat vitae, eleifend ac, enim. Aliquam lorem ante, dapibus in, viverra quis, feugiat a, tellus. Phasellus viverra nulla ut metus varius laoreet. Quisque rutrum. Aenean imperdiet. Etiam ultricies nisi vel augue. Curabitur ullamcorper ultricies

\section{Risultati}
Lorem ipsum dolor sit amet, consectetuer adipiscing elit. Aenean commodo ligula eget dolor. Aenean massa. Cum sociis natoque penatibus et magnis dis parturient montes, nascetur ridiculus mus. Donec quam felis, ultricies nec, pellentesque eu, pretium quis, sem. Nulla consequat massa quis enim. Donec pede justo, fringilla vel, aliquet nec, vulputate eget, arcu. In enim justo, rhoncus ut, imperdiet a, venenatis vitae, justo. Nullam dictum felis eu pede mollis pretium. Integer tincidunt. Cras dapibus. Vivamus elementum semper nisi. Aenean vulputate eleifend tellus. Aenean leo ligula, porttitor eu, consequat vitae, eleifend ac, enim. Aliquam lorem ante, dapibus in, viverra quis, feugiat a, tellus. Phasellus viverra nulla ut metus varius laoreet. Quisque rutrum. Aenean imperdiet. Etiam ultricies nisi vel augue. Curabitur ullamcorper ultricies

\section{Dashboard}
Lorem ipsum dolor sit amet, consectetuer adipiscing elit. Aenean commodo ligula eget dolor. Aenean massa. Cum sociis natoque penatibus et magnis dis parturient montes, nascetur ridiculus mus. Donec quam felis, ultricies nec, pellentesque eu, pretium quis, sem. Nulla consequat massa quis enim. Donec pede justo, fringilla vel, aliquet nec, vulputate eget, arcu. In enim justo, rhoncus ut, imperdiet a, venenatis vitae, justo. Nullam dictum felis eu pede mollis pretium. Integer tincidunt. Cras dapibus. Vivamus elementum semper nisi. Aenean vulputate eleifend tellus. Aenean leo ligula, porttitor eu, consequat vitae, eleifend ac, enim. Aliquam lorem ante, dapibus in, viverra quis, feugiat a, tellus. Phasellus viverra nulla ut metus varius laoreet. Quisque rutrum. Aenean imperdiet. Etiam ultricies nisi vel augue. Curabitur ullamcorper ultricies

\section{Conclusioni}
Lorem ipsum dolor sit amet, consectetuer adipiscing elit. Aenean commodo ligula eget dolor. Aenean massa. Cum sociis natoque penatibus et magnis dis parturient montes, nascetur ridiculus mus. Donec quam felis, ultricies nec, pellentesque eu, pretium quis, sem. Nulla consequat massa quis enim. Donec pede justo, fringilla vel, aliquet nec, vulputate eget, arcu. In enim justo, rhoncus ut, imperdiet a, venenatis vitae, justo. Nullam dictum felis eu pede mollis pretium. Integer tincidunt. Cras dapibus. Vivamus elementum semper nisi. Aenean vulputate eleifend tellus. Aenean leo ligula, porttitor eu, consequat vitae, eleifend ac, enim. Aliquam lorem ante, dapibus in, viverra quis, feugiat a, tellus. Phasellus viverra nulla ut metus varius laoreet. Quisque rutrum. Aenean imperdiet. Etiam ultricies nisi vel augue. Curabitur ullamcorper ultricies

\begin{align}
	A = 
	\begin{bmatrix}
	A_{11} & A_{21} \\
  	A_{21} & A_{22}
	\end{bmatrix}
\end{align}

\printbibliography[title={Bibliografia}] %Prints bibliography

\end{document}