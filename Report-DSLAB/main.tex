%%%%%%%%%%%%%%%%%%%%%%%%%%%%%%%%%%%%%%%%%
% Wenneker Article
% LaTeX Template
% Version 2.0 (28/2/17)
%
% This template was downloaded from:
% http://www.LaTeXTemplates.com
%
% Authors:
% Vel (vel@LaTeXTemplates.com)
% Frits Wenneker
%
% License:
% CC BY-NC-SA 3.0 (http://creativecommons.org/licenses/by-nc-sa/3.0/)
%
%%%%%%%%%%%%%%%%%%%%%%%%%%%%%%%%%%%%%%%%%

%----------------------------------------------------------------------------------------
%	PACKAGES AND OTHER DOCUMENT CONFIGURATIONS
%----------------------------------------------------------------------------------------

\documentclass[12pt, a4paper, twocolumn]{article} % 10pt font size (11 and 12 also possible), A4 paper (letterpaper for US letter) and two column layout (remove for one column)
\usepackage{setspace}
\usepackage{multirow}
\usepackage{float}
\usepackage{hyperref}
\usepackage[USenglish,UKenglish,french,spanish,italian]{babel}

\input{structure.tex} % Specifies the document structure and loads requires packages

%----------------------------------------------------------------------------------------
%	ARTICLE INFORMATION
%----------------------------------------------------------------------------------------

\title{Data Science Lab: Vendite E-commerce} % The article title

\author{
	Davide Abete x\\
	Fabrizio Cominetti x\\
	Ruben Agazzi x\\
	Tommaso Strada x\\
	Alessandro Fasani x % Authors
}

%----------------------------------------------------------------------------------------

\begin{document}

\selectlanguage{italian}

% \maketitle % Print the title

\thispagestyle{firstpage} % Apply the page style for the first page (no headers and footers)

%----------------------------------------------------------------------------------------
%	ABSTRACT
%----------------------------------------------------------------------------------------
\twocolumn[
  \begin{@twocolumnfalse}
    \maketitle
    \begin{abstract}\\
      \noindent{L'obiettivo principale del progetto è quello di valutare e selezionare il miglior modello predittivo relativamente alle stime delle vendite di alcuni settori, precisamente Pesca, Calcio e x, di un attività di e-commerce.}\\
      Il periodo preso in considerazione va dal giorno x al giorno x.\\
      Per un attività commerciale presente sulla rete è di fondamentale importanza prevedere le vendite che saranno effettuate nei periodi successivi per ogni settore a disposizione, in modo tale da organizzare disponibilità di magazzino e di spedizione.\\
      I dati sono stati modellati in funzione dell'obiettivo e aggregati secondo diverse granularità, ovvero con frequenza annuale, trimestrale, mensile e settimanale, in modo tale da valutare le performance dei modelli presi in considerazione.\\
      Come sarà possibile verificare nel proseguio del report, la granularità più efficace, relativamente agli obiettivi preposti dal progetto, è risultata quella x.\\
      Il progetto è stato effettuato prendendo in considerazione i tre settori con maggiore disponibilità effettiva di osservazioni, ma i modelli utilizzati possono essere applicati anche ad altri settori, per realizzare in questo modo una panoramica completo sull'intero e-commerce.
    \end{abstract}
  \end{@twocolumnfalse}
]
\bigskip

\clearpage
% \lettrineabstract{x}
\bigskip
\tableofcontents

%----------------------------------------------------------------------------------------
%	ARTICLE CONTENTS
%----------------------------------------------------------------------------------------

\section{Introduzione}
L'obiettivo principale del progetto è quello di valutare e selezionare il miglior modello predittivo relativamente alle stime delle vendite di alcuni settori, precisamente Pesca, Calcio e x, di un attività di e-commerce.
Il periodo preso in considerazione va dal giorno x al giorno x.
Per un attività commerciale presente sulla rete è di fondamentale importanza prevedere le vendite che saranno effettuate nei periodi successivi per ogni settore a disposizione, in modo tale da organizzare disponibilità di magazzino e di spedizione.

I dati sono stati modellati in funzione dell'obiettivo e aggregati secondo diverse granularità, ovvero con frequenza annuale, trimestrale, mensile e settimanale, in modo tale da valutare le performance dei modelli presi in considerazione.

Come sarà possibile verificare nel proseguio del report, la granularità più efficace, relativamente agli obiettivi preposti dal progetto, è risultata quella x.

Il progetto è stato effettuato prendendo in considerazione i tre settori con maggiore disponibilità effettiva di osservazioni, ma i modelli utilizzati possono essere applicati anche ad altri settori, per realizzare in questo modo una panoramica completo sull'intero e-commerce.

\subsection{Punti Principali}
\begin{itemize}
	\item Vendite 
	\item Settori 
	\item Previsione
\end{itemize}

\subsection*{Heading on level 2}
Lorem ipsum dolor sit amet, consectetuer adipiscing elit. Aenean commodo ligula eget dolor. Aenean massa. Cum sociis natoque penatibus et magnis dis parturient montes, nascetur ridiculus mus. Donec quam felis, ultricies nec, pellentesque eu, pretium quis, sem.
\bigskip

\begin{itemize}
	\item First item in a list 
	\item Second item in a list 
	\item Third item in a list
\end{itemize}
Lorem ipsum dolor sit amet, consectetuer adipiscing elit. Aenean commodo ligula eget dolor. Aenean massa. Cum sociis natoque penatibus et magnis dis parturient montes, nascetur ridiculus mus. Donec quam felis, ultricies nec, pellentesque eu, pretium quis, sem. Nulla consequat massa quis enim. 

Donec pede justo, fringilla vel, aliquet nec, vulputate eget, arcu. In enim justo, rhoncus ut, imperdiet a, venenatis vitae, justo. Nullam dictum felis eu pede mollis pretium. Integer tincidunt. Cras dapibus. Vivamus elementum semper nisi. Aenean vulputate eleifend tellus. Aenean leo ligula, porttitor eu, consequat vitae, eleifend ac, enim. Aliquam lorem ante, dapibus in, viverra quis, feugiat a, tellus. Phasellus viverra nulla ut metus varius laoreet. Quisque rutrum. Aenean imperdiet. Etiam ultricies nisi vel augue. Curabitur ullamcorper ultricies 

\section{Obiettivo}
Lorem ipsum dolor sit amet, consectetuer adipiscing elit. Aenean commodo ligula eget dolor. Aenean massa. Cum sociis natoque penatibus et magnis dis parturient montes, nascetur ridiculus mus. Donec quam felis, ultricies nec, pellentesque eu, pretium quis, sem. Nulla consequat massa quis enim. Donec pede justo, fringilla vel, aliquet nec, vulputate eget, arcu. In enim justo, rhoncus ut, imperdiet a, venenatis vitae, justo. Nullam dictum felis eu pede mollis pretium. Integer tincidunt. Cras dapibus. Vivamus elementum semper nisi. Aenean vulputate eleifend tellus. Aenean leo ligula, porttitor eu, consequat vitae, eleifend ac, enim. Aliquam lorem ante, dapibus in, viverra quis, feugiat a, tellus. Phasellus viverra nulla ut metus varius laoreet. Quisque rutrum. Aenean imperdiet. Etiam ultricies nisi vel augue. Curabitur ullamcorper ultricies 

\begin{table}
\caption{Random table}
\centering
	\begin{tabular}{llr}
		\toprule
		\multicolumn{2}{c}{Name} \\
		\cmidrule(r){1-2}
			First name & Last Name & Grade \\
		\midrule
			John & Doe & $7.5$ \\
			Richard & Miles & $2$ \\
		\bottomrule
	\end{tabular}
\end{table}

\section{Aspetti Metodologici}
I modelli selezionati ed utilizzati all'interno del progetto sono i seguenti: ARIMA, TBATS, PROPHET, XGBOOST.\\
Di seguito osserviamo i modelli in modo più approfondito da un punto di vista teorico.

\subsection*{ARIMA}
Lorem ipsum dolor sit amet, consectetuer adipiscing elit. Aenean commodo ligula eget dolor. Aenean massa. Cum sociis natoque penatibus et magnis dis parturient montes, nascetur ridiculus mus. Donec quam felis, ultricies nec, pellentesque eu, pretium quis, sem. 

\begin{description}
	\item[First] This is the first item
	\item[Last] This is the last item
\end{description}
Nullam dictum felis eu pede mollis pretium. Integer tincidunt. Cras dapibus. Vivamus elementum semper nisi. Aenean vulputate eleifend tellus. Aenean leo ligula, porttitor eu, consequat vitae, eleifend ac, enim. Aliquam lorem ante, dapibus in, viverra quis, feugiat a, tellus. Phasellus viverra nulla ut metus varius laoreet. Quisque rutrum. Aenean imperdiet. Etiam ultricies nisi vel augue. Curabitur ullamcorper ultricies

\subsection*{TBATS}
Lorem ipsum dolor sit amet, consectetuer adipiscing elit. Aenean commodo ligula eget dolor. Aenean massa. Cum sociis natoque penatibus et magnis dis parturient montes, nascetur ridiculus mus. Donec quam felis, ultricies nec, pellentesque eu, pretium quis, sem. 

\begin{description}
	\item[First] This is the first item
	\item[Last] This is the last item
\end{description}
Nullam dictum felis eu pede mollis pretium. Integer tincidunt. Cras dapibus. Vivamus elementum semper nisi. Aenean vulputate eleifend tellus. Aenean leo ligula, porttitor eu, consequat vitae, eleifend ac, enim. Aliquam lorem ante, dapibus in, viverra quis, feugiat a, tellus. Phasellus viverra nulla ut metus varius laoreet. Quisque rutrum. Aenean imperdiet. Etiam ultricies nisi vel augue. Curabitur ullamcorper ultricies

\subsection*{PROPHET}
Lorem ipsum dolor sit amet, consectetuer adipiscing elit. Aenean commodo ligula eget dolor. Aenean massa. Cum sociis natoque penatibus et magnis dis parturient montes, nascetur ridiculus mus. Donec quam felis, ultricies nec, pellentesque eu, pretium quis, sem. 

\begin{description}
	\item[First] This is the first item
	\item[Last] This is the last item
\end{description}
Nullam dictum felis eu pede mollis pretium. Integer tincidunt. Cras dapibus. Vivamus elementum semper nisi. Aenean vulputate eleifend tellus. Aenean leo ligula, porttitor eu, consequat vitae, eleifend ac, enim. Aliquam lorem ante, dapibus in, viverra quis, feugiat a, tellus. Phasellus viverra nulla ut metus varius laoreet. Quisque rutrum. Aenean imperdiet. Etiam ultricies nisi vel augue. Curabitur ullamcorper ultricies

\subsection*{XGBOOST}
Lorem ipsum dolor sit amet, consectetuer adipiscing elit. Aenean commodo ligula eget dolor. Aenean massa. Cum sociis natoque penatibus et magnis dis parturient montes, nascetur ridiculus mus. Donec quam felis, ultricies nec, pellentesque eu, pretium quis, sem. 

\begin{description}
	\item[First] This is the first item
	\item[Last] This is the last item
\end{description}
Nullam dictum felis eu pede mollis pretium. Integer tincidunt. Cras dapibus. Vivamus elementum semper nisi. Aenean vulputate eleifend tellus. Aenean leo ligula, porttitor eu, consequat vitae, eleifend ac, enim. Aliquam lorem ante, dapibus in, viverra quis, feugiat a, tellus. Phasellus viverra nulla ut metus varius laoreet. Quisque rutrum. Aenean imperdiet. Etiam ultricies nisi vel augue. Curabitur ullamcorper ultricies

\section{Dati}
Lorem ipsum dolor sit amet, consectetuer adipiscing elit. Aenean commodo ligula eget dolor. Aenean massa. Cum sociis natoque penatibus et magnis dis parturient montes, nascetur ridiculus mus. Donec quam felis, ultricies nec, pellentesque eu, pretium quis, sem. Nulla consequat massa quis enim. Donec pede justo, fringilla vel, aliquet nec, vulputate eget, arcu. In enim justo, rhoncus ut, imperdiet a, venenatis vitae, justo. Nullam dictum felis eu pede mollis pretium. Integer tincidunt. Cras dapibus. Vivamus elementum semper nisi. Aenean vulputate eleifend tellus. Aenean leo ligula, porttitor eu, consequat vitae, eleifend ac, enim. Aliquam lorem ante, dapibus in, viverra quis, feugiat a, tellus. Phasellus viverra nulla ut metus varius laoreet. Quisque rutrum. Aenean imperdiet. Etiam ultricies nisi vel augue. Curabitur ullamcorper ultricies

\subsection{Manipolazione Dati}
Lorem ipsum dolor sit amet, consectetuer adipiscing elit. Aenean commodo ligula eget dolor. Aenean massa. Cum sociis natoque penatibus et magnis dis parturient montes, nascetur ridiculus mus. Donec quam felis, ultricies nec, pellentesque eu, pretium quis, sem. 

\section{Analisi Pesca}
Lorem ipsum dolor sit amet, consectetuer adipiscing elit. Aenean commodo ligula eget dolor. Aenean massa. Cum sociis natoque penatibus et magnis dis parturient montes, nascetur ridiculus mus. Donec quam felis, ultricies nec, pellentesque eu, pretium quis, sem. Nulla consequat massa quis enim. Donec pede justo, fringilla vel, aliquet nec, vulputate eget, arcu. In enim justo, rhoncus ut, imperdiet a, venenatis vitae, justo. Nullam dictum felis eu pede mollis pretium. Integer tincidunt. Cras dapibus. Vivamus elementum semper nisi. Aenean vulputate eleifend tellus. Aenean leo ligula, porttitor eu, consequat vitae, eleifend ac, enim. Aliquam lorem ante, dapibus in, viverra quis, feugiat a, tellus. Phasellus viverra nulla ut metus varius laoreet. Quisque rutrum. Aenean imperdiet. Etiam ultricies nisi vel augue. Curabitur ullamcorper ultricies

\section{Analisi Calcio}
Lorem ipsum dolor sit amet, consectetuer adipiscing elit. Aenean commodo ligula eget dolor. Aenean massa. Cum sociis natoque penatibus et magnis dis parturient montes, nascetur ridiculus mus. Donec quam felis, ultricies nec, pellentesque eu, pretium quis, sem. Nulla consequat massa quis enim. Donec pede justo, fringilla vel, aliquet nec, vulputate eget, arcu. In enim justo, rhoncus ut, imperdiet a, venenatis vitae, justo. Nullam dictum felis eu pede mollis pretium. Integer tincidunt. Cras dapibus. Vivamus elementum semper nisi. Aenean vulputate eleifend tellus. Aenean leo ligula, porttitor eu, consequat vitae, eleifend ac, enim. Aliquam lorem ante, dapibus in, viverra quis, feugiat a, tellus. Phasellus viverra nulla ut metus varius laoreet. Quisque rutrum. Aenean imperdiet. Etiam ultricies nisi vel augue. Curabitur ullamcorper ultricies

\section{Risultati}
Lorem ipsum dolor sit amet, consectetuer adipiscing elit. Aenean commodo ligula eget dolor. Aenean massa. Cum sociis natoque penatibus et magnis dis parturient montes, nascetur ridiculus mus. Donec quam felis, ultricies nec, pellentesque eu, pretium quis, sem. Nulla consequat massa quis enim. Donec pede justo, fringilla vel, aliquet nec, vulputate eget, arcu. In enim justo, rhoncus ut, imperdiet a, venenatis vitae, justo. Nullam dictum felis eu pede mollis pretium. Integer tincidunt. Cras dapibus. Vivamus elementum semper nisi. Aenean vulputate eleifend tellus. Aenean leo ligula, porttitor eu, consequat vitae, eleifend ac, enim. Aliquam lorem ante, dapibus in, viverra quis, feugiat a, tellus. Phasellus viverra nulla ut metus varius laoreet. Quisque rutrum. Aenean imperdiet. Etiam ultricies nisi vel augue. Curabitur ullamcorper ultricies

\section{Dashboard}
Lorem ipsum dolor sit amet, consectetuer adipiscing elit. Aenean commodo ligula eget dolor. Aenean massa. Cum sociis natoque penatibus et magnis dis parturient montes, nascetur ridiculus mus. Donec quam felis, ultricies nec, pellentesque eu, pretium quis, sem. Nulla consequat massa quis enim. Donec pede justo, fringilla vel, aliquet nec, vulputate eget, arcu. In enim justo, rhoncus ut, imperdiet a, venenatis vitae, justo. Nullam dictum felis eu pede mollis pretium. Integer tincidunt. Cras dapibus. Vivamus elementum semper nisi. Aenean vulputate eleifend tellus. Aenean leo ligula, porttitor eu, consequat vitae, eleifend ac, enim. Aliquam lorem ante, dapibus in, viverra quis, feugiat a, tellus. Phasellus viverra nulla ut metus varius laoreet. Quisque rutrum. Aenean imperdiet. Etiam ultricies nisi vel augue. Curabitur ullamcorper ultricies

\section{Conclusioni}
Lorem ipsum dolor sit amet, consectetuer adipiscing elit. Aenean commodo ligula eget dolor. Aenean massa. Cum sociis natoque penatibus et magnis dis parturient montes, nascetur ridiculus mus. Donec quam felis, ultricies nec, pellentesque eu, pretium quis, sem. Nulla consequat massa quis enim. Donec pede justo, fringilla vel, aliquet nec, vulputate eget, arcu. In enim justo, rhoncus ut, imperdiet a, venenatis vitae, justo. Nullam dictum felis eu pede mollis pretium. Integer tincidunt. Cras dapibus. Vivamus elementum semper nisi. Aenean vulputate eleifend tellus. Aenean leo ligula, porttitor eu, consequat vitae, eleifend ac, enim. Aliquam lorem ante, dapibus in, viverra quis, feugiat a, tellus. Phasellus viverra nulla ut metus varius laoreet. Quisque rutrum. Aenean imperdiet. Etiam ultricies nisi vel augue. Curabitur ullamcorper ultricies

\begin{align}
	A = 
	\begin{bmatrix}
	A_{11} & A_{21} \\
  	A_{21} & A_{22}
	\end{bmatrix}
\end{align}

\printbibliography[title={Bibliografia}] %Prints bibliography

\end{document}